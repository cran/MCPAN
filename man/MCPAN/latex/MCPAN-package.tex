\HeaderA{MCPAN-package}{Multiple comparison procedures using normal approximation}{MCPAN.Rdash.package}
\aliasA{MCPAN}{MCPAN-package}{MCPAN}
\keyword{package}{MCPAN-package}
\begin{Description}\relax
Multiple contrast tests and simultaneous confidence intervals using normal approximation, if individuals
are randomly assigned to treatments in a oneway layout. If the variable of interest is dichotomous, the binom-methods can be used.
If the variable of interest is the rate of tumours in long-term rodent carcinogenicity trials (without cause of death information), the poly3-methods can be used.
The methods implemented in this package are NOT published in peer-reviewed journals so far.
\end{Description}
\begin{Details}\relax
\Tabular{ll}{
Package: & MCPAN\\
Type: & Package\\
Version: & 1.0-4\\
Date: & 2007-08-15\\
License: & GPL \\
}
\end{Details}
\begin{Author}\relax
Frank Schaarschmidt, Daniel Gerhard
Maintainer: Frank Schaarschmidt <schaarschmidt@biostat.uni-hannover.de>
\end{Author}
\begin{References}\relax
For long-term rodent carcinogenicity data: The assumptions of poly-3-adjustment are outlined in:

Bailer, J.A. and Portier, C.J. (1988):
Effects of treatment-induced mortality and tumor-induced mortality on tests for carcinogenicity in small samples.
Biometrics 44, 417-431.

Peddada, S.D., Dinse, G.E., and Haseman, J.K. (2005):
A survival-adjusted quantal response test for comparing tumor incidence rates.
Applied Statistics 54, 51-61.

For correlation of multiple contrasts of binomial proportions, see:
Bretz F, Hothorn L.:
Detecting dose-response using contrasts: asymptotic power and sample size determination for binomial data.
Statistics in Medicine 2002; 21: 3325-3335.

Simulation results (coverage probability of simultaneous confidence intervals)
for the binomial proportions and poly-3-adjusted tumour rates can be found in:

Sill, M. (2007):
....
Master thesis, Institute of Biostatistics, Leibniz University Hannover.
\end{References}
\begin{SeeAlso}\relax
\end{SeeAlso}
\begin{Examples}
\begin{ExampleCode}

# # # 1)
# Adjusted p-values and simultaneous confidence intervals 
# for 2xk tables of binomial data: 
# binomtest, binomci

# Difference of proportions

binomRDtest(x=c(2,6,4,13), n=c(34,33,36,34),
 names=c("Placebo", "50", "75", "150"),
 type="Dunnett", method="ADD1")

binomRDci(x=c(2,6,4,13), n=c(34,33,36,34),
 names=c("Placebo", "50", "75", "150"),
 type="Dunnett", method="ADD1")

# Odds ratios:

binomORci(x=c(2,6,4,13), n=c(34,33,36,34),
 names=c("Placebo", "50", "75", "150"),
 type="Dunnett")

# For more details on evaluation,
# see:
# ?liarozole

 data(liarozole) 


# # # 2)
# Adjusted p-values and simultaneous confidence intervals 
# for poly-3-adjusted tumour rates: 
# poly3test, poly3ci

data(methyl)
methyl

# poly-3-adjusted sample estimates:

poly3estf(time=methyl$death,
 status=methyl$tumour,
 f=methyl$group)

# Simultaneous Add-1-confidence intervals
# for difference to the control group:

poly3ci(time=methyl$death, status=methyl$tumour,
 f=methyl$group, method="ADD1",
 type="Dunnett", alternative="greater")

# Test for trend, based on Changepoint contrasts:

poly3test(time=methyl$death, status=methyl$tumour,
 f=methyl$group, method="ADD1",
 type="Changepoint", alternative="greater")

# # # 3) Plot of confidence intervals
# created by binomci and poly3ci:

MethylCI <- poly3ci(time=methyl$death, status=methyl$tumour,
 f=methyl$group, method="ADD1",
 type="Dunnett", alternative="greater")

plot(MethylCI)



\end{ExampleCode}
\end{Examples}

