\HeaderA{Waldtest}{Simultaneous Wald tests}{Waldtest}
\keyword{htest}{Waldtest}
\begin{Description}\relax
General function for adjusted p-values for an UIT in a one-way layout using multivariate normal distribution.
\end{Description}
\begin{Usage}
\begin{verbatim}
Waldtest(estp, varp, cmat, alternative = "greater")
\end{verbatim}
\end{Usage}
\begin{Arguments}
\begin{ldescription}
\item[\code{estp}] numeric vector of point estimates of length I, with I = the number of samples
\item[\code{varp}] numeric vector of variance estimates of length I, to be used for interval construction 
\item[\code{cmat}] Contrast matrix of dimension MxI, with M = the number of contrasts, I= the number of samples 
\item[\code{alternative}] character string 
\end{ldescription}
\end{Arguments}
\begin{Details}\relax
\end{Details}
\begin{Value}
A list containing:
\begin{ldescription}
\item[\code{teststat}] a numeric vector of teststatistics of length M
\item[\code{pval}] a single numeric p-value, the p-value of the maximum test (minimum p-value)
\item[\code{p.val.adj}] a vector of length M, the adjusted p-values of the single contrasts
\item[\code{alternative}] a single character vector, as the input
\end{ldescription}
\end{Value}
\begin{Author}\relax
Frank Schaarschmidt
\end{Author}
\begin{SeeAlso}\relax
For user level implementations see:

\code{\LinkA{binomRDtest}{binomRDtest}},
\code{\LinkA{poly3test}{poly3test}}
\end{SeeAlso}

