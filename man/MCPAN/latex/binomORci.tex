\HeaderA{binomORci}{Simultaneous confidence intervals for odds ratios}{binomORci}
\methaliasA{binomORci.default}{binomORci}{binomORci.default}
\methaliasA{binomORci.formula}{binomORci}{binomORci.formula}
\methaliasA{binomORci.matrix}{binomORci}{binomORci.matrix}
\methaliasA{binomORci.table}{binomORci}{binomORci.table}
\keyword{htest}{binomORci}
\begin{Description}\relax
Approximate simultaneous confidence intervals for (weighted geometric means of) odds ratios
are constructed. Estimates are derived from fitting a glm on the logit-link,
approximate intervals are constructed on the log-link,
and transformed to origninal scale.
\end{Description}
\begin{Usage}
\begin{verbatim}
binomORci(x, ...)

## Default S3 method:
binomORci(x, n, names = NULL,
 type = "Dunnett", cmat = NULL, alternative = "two.sided",
 conf.level = 0.95, ...)

## S3 method for class 'formula':
binomORci(formula, data,
 type = "Dunnett", cmat = NULL, alternative = "two.sided",
 conf.level = 0.95, ...)

## S3 method for class 'table':
binomORci(x,
 type = "Dunnett", cmat = NULL, alternative = "two.sided",
 conf.level = 0.95, ...)

## S3 method for class 'matrix':
binomORci(x,
 type = "Dunnett", cmat = NULL, alternative = "two.sided",
 conf.level = 0.95, ...)
\end{verbatim}
\end{Usage}
\begin{Arguments}
\begin{ldescription}
\item[\code{x}] a numeric vector, giving the number of successes in I independent samples,
or an object of class \code{"table"}, representing the 2xk-table,
or an object of class \code{"matrix"}, representing the 2xk-table
\item[\code{n}] numeric vector, giving the number of trials (i.e. the sample size) in each of the I groups
(only required if \code{x} is a numeric vector, ignored otherwise)  
\item[\code{names}] an optional character string, giving the names of the groups/ sample in \code{x}, \code{n};
if not specified the possible names of x are taken as group names (ignored if \code{x} is a table or matrix)
\item[\code{formula}] a two-sided formula of the style 'response ~ treatment', where 'response' should be a categorical variable with two levels,
while treatment should be a factor specifying the treatment levels
\item[\code{data}] a data.frame, containing the variables specified in formula
\item[\code{type}] a character string, giving the name of a contrast method,
as defined in \code{contrMat(multcomp)}; ignored if \code{cmat} is sepcified 
\item[\code{cmat}] a optional contrast matrix 
\item[\code{alternative}] a single character string, one of "two.sided", "less", "greater" 
\item[\code{conf.level}] a single numeric value, simultaneous confidence level 
\item[\code{...}] arguments to be passed to \code{\LinkA{binomest}{binomest}}, currently only \code{success} labelling the event which should be considered as success
\end{ldescription}
\end{Arguments}
\begin{Details}\relax
\end{Details}
\begin{Value}
A object of class "binomORci", a list containing:
\begin{ldescription}
\item[\code{conf.int}] a matrix with 2 columns: lower and upper confidence bounds, and M rows
\item[\code{alternative }] character string, as input
\item[\code{conf.level}] single numeric value, as input
\item[\code{estimate}] a matrix with 1 column: containing the estimates of the contrasts
\item[\code{x}] the observed number of successes
\item[\code{n}] the number of trials
\item[\code{p}] the estimated proportions
\item[\code{success}] a character string labelling the event considered as success
\item[\code{names}] the group names
\item[\code{method}] a character string, specifying the method of interval construction
\item[\code{cmat}] the contrast matrix used
\end{ldescription}
\end{Value}
\begin{Note}\relax
\end{Note}
\begin{Author}\relax
Daniel Gerhard, Frank Schaarschmidt
\end{Author}
\begin{References}\relax
. . .
\end{References}
\begin{SeeAlso}\relax
Intervals for the risk difference \code{\LinkA{binomRDci}{binomRDci}},
summary for odds ratio confidence intervals \code{\LinkA{summary.binomORci}{summary.binomORci}}
plot for confidence intervals \code{\LinkA{plot.sci}{plot.sci}}
\end{SeeAlso}
\begin{Examples}
\begin{ExampleCode}
data(liarozole)

table(liarozole)

ORlia<-binomORci(Improved ~ Treatment, data=liarozole, success="y")
ORlia
summary(ORlia)
plot(ORlia)

# if data are available as table:

tab<-table(liarozole)
ORlia2<-binomORci(tab, success="y")
ORlia2


\end{ExampleCode}
\end{Examples}

