\HeaderA{censsample}{Random data for Poly-k}{censsample}
\keyword{datagen}{censsample}
\begin{Description}\relax
Random numbers from two independent Weibull distributions for Mortality and tumour induction.
\end{Description}
\begin{Usage}
\begin{verbatim}
censsample(n, scale.m, shape.m, scale.t, shape.t = 3, tmax)
\end{verbatim}
\end{Usage}
\begin{Arguments}
\begin{ldescription}
\item[\code{n}] a single numeric value, the number of individuals 
\item[\code{scale.m}] a single numeric value, scale parameter of the Weibull distribution for mortality 
\item[\code{shape.m}] a single numeric value, shape parameter of the Weibull distribution for mortality 
\item[\code{scale.t}] a single numeric value, scale parameter of the Weibull distribution for tumour induction 
\item[\code{shape.t}] a single numeric value, shape parameter of the Weibull distribution for tumour induction 
\item[\code{tmax}] a single numeric value, maximum time in the trial 
\end{ldescription}
\end{Arguments}
\begin{Details}\relax
\end{Details}
\begin{Value}
A data.frame with columns
\begin{ldescription}
\item[\code{time }] a numeric vector of length \code{n}, the time of death of an individual
\item[\code{status }] a logical vector of length \code{n}, the tumour status at time of death (TRUE: tumour present, FALSE: no tumour present)
\item[\code{T.t}] time of tumour induction (unobservable)
\item[\code{T.m}] time of death
\item[\code{tmax}] maximum time of death
\end{ldescription}
\end{Value}
\begin{Author}\relax
Frank Schaarschmidt
\end{Author}
\begin{References}\relax
Peddada (2005), Johnson, Kotz,...
\end{References}

