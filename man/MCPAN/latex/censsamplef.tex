\HeaderA{censsamplef}{Random data for Poly-k}{censsamplef}
\keyword{datagen}{censsamplef}
\begin{Description}\relax
Random data for Poly-k for a one-way layout, with I groups.
\end{Description}
\begin{Usage}
\begin{verbatim}
censsamplef(n, scale.m, shape.m, scale.t, shape.t = 3, tmax)
\end{verbatim}
\end{Usage}
\begin{Arguments}
\begin{ldescription}
\item[\code{n}] a numeric vector, the numbers of individuals of length I 
\item[\code{scale.m}] a numeric vector, scale parameters of the Weibull distribution for mortality 
\item[\code{shape.m}] a numeric vector, shape parameters of the Weibull distribution for mortality 
\item[\code{scale.t}] a numeric vector, scale parameters of the Weibull distribution for tumour induction 
\item[\code{shape.t}] a numeric vector, shape parameters of the Weibull distribution for tumour induction 
\item[\code{tmax}] a single numeric value, maximum time in the trial 
\end{ldescription}
\end{Arguments}
\begin{Details}\relax
\end{Details}
\begin{Value}
A data.frame with columns
\begin{ldescription}
\item[\code{time}] a numeric vector of length \code{n}, the time of death of an individual
\item[\code{status}] a logical vector of length \code{n}, the tumour status at time of death (TRUE: tumour present, FALSE: no tumour present)
\item[\code{T.t}] time of tumour induction (unobservable)
\item[\code{T.m}] time of death
\item[\code{tmax}] maximum time of death
\item[\code{f}] a factor of containing an appropriate grouping variable
\end{ldescription}
\end{Value}
\begin{Author}\relax
Frank Schaarschmidt
\end{Author}

