\HeaderA{powerbinom}{Approximate power for multiple contrast tests of binomial proportions}{powerbinom}
\keyword{htest}{powerbinom}
\begin{Description}\relax
Approximative power calculation for multiple contrast tests of binomial proportions, based on probabilities of the multivariate standard normal distribution.
\end{Description}
\begin{Usage}
\begin{verbatim}
powerbinom(p, n, alpha = 0.05, type = "Williams", cmat = NULL, method = "Wald", alternative = "less")
\end{verbatim}
\end{Usage}
\begin{Arguments}
\begin{ldescription}
\item[\code{p}] a numeric vector, the proportions assumed under the alternative 
\item[\code{n}] an integer vector, the number of observations, must be of same length as \code{p} 
\item[\code{alpha}] a single numeric value, the alpha-level of the test  
\item[\code{type}] if \code{cmat} is not specified, the type of multiple contrast test to be applied on the proportions \code{p}; will be ignored if \code{cmat} is specified
\item[\code{cmat}] a numeric contrast matrix, number of columns must be the same as length of \code{n}, and \code{p}; if secified, \code{type} will be ignored 
\item[\code{method}] a character string, the method for variance estimation in test / confidence interval construction: one of "Wald", "ADD1", "ADD2"; see \code{\LinkA{binomRDtest}{binomRDtest}} 
\item[\code{alternative}] a character string, specifying the alternative, options are "two.sided", "less", "greater" 
\end{ldescription}
\end{Arguments}
\begin{Details}\relax
Calls \code{\LinkA{powermcp}{powermcp}}. 

Power for an Union-Intersection-Test is calculated, i.e. the power that H0 is rejected for at least one of the contrasts specified by \code{cmat} or \code{type}.

Note, that currently only power calculation for Union-Intersection-Tests of 
type H0: ALL Lm<=0 vs.  HA: ANY Lm>0 (alternative="greater"), H0: ALL Lm>=0 vs.  HA: ANY Lm<0 (alternative="less"), or H0: ALL Lm=0 vs.  HA: ANY Lm<or>0 (alternative="two.sided") are implemented.
Power for tests against delta>0 or delta<0 is not implemented.

Calcualtion is based on probabilities of the multivariate standard normal distribution, thus will be only appropriate if normal approximation for the test statistics holds.
Therefore, results might be misleading for small sample sizes, such as n < 40, and a warning is posted. In this case, simulation is recommended.
\end{Details}
\begin{Value}
A single numeric value, the power.
\end{Value}
\begin{Note}\relax
\end{Note}
\begin{Author}\relax
Frank Schaarschmidt
\end{Author}
\begin{References}\relax
\end{References}
\begin{SeeAlso}\relax
\code{\LinkA{powermcp}{powermcp}}
\end{SeeAlso}
\begin{Examples}
\begin{ExampleCode}

# Assume, one wants to perform a test for increasing trend
#  using Williams type of contrasts among I=5 groups
#  (e.g. 4 doses and one control).
#  Proportions are assumed to have values
#  pi=(0.1,0.12,0.14,0.14,0.2) under the alternative.


powerbinom(p=c(0.1, 0.12, 0.14, 0.14, 0.2),
 n=c(20,20,20,20,20), type = "Williams",
  method = "ADD1", alternative = "greater")

powerbinom(p=c(0.1, 0.12, 0.14, 0.14, 0.2),
 n=c(30,30,30,30,30), type = "Williams",
  method = "ADD1", alternative = "greater")

powerbinom(p=c(0.1, 0.12, 0.14, 0.14, 0.2),
 n=c(60,60,60,60,60), type = "Williams",
  method = "ADD1", alternative = "greater")

powerbinom(p=c(0.1, 0.12, 0.14, 0.14, 0.2),
 n=c(80,80,80,80,80), type = "Williams",
  method = "ADD1", alternative = "greater")

powerbinom(p=c(0.1, 0.12, 0.14, 0.14, 0.2),
 n=c(100,100,100,100,100), type = "Williams",
  method = "ADD1", alternative = "greater")

powerbinom(p=c(0.1, 0.12, 0.14, 0.14, 0.2),
 n=c(150,150,150,150,150), type = "Williams",
  method = "ADD1", alternative = "greater")

powerbinom(p=c(0.1, 0.12, 0.14, 0.14, 0.2),
 n=c(190,140,140,140,140), type = "Williams",
  method = "ADD1", alternative = "greater")

\end{ExampleCode}
\end{Examples}

