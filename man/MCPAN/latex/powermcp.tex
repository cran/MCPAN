\HeaderA{powermcp}{Approximative power calculation for multiple contrast tests}{powermcp}
\keyword{htest}{powermcp}
\begin{Description}\relax
Approximative power calculation for multiple contrast tests, based on normal approximation.
\end{Description}
\begin{Usage}
\begin{verbatim}
powermcp(TExpH1, alpha = 0.05, corrmatH1, alternative = "two.sided")
\end{verbatim}
\end{Usage}
\begin{Arguments}
\begin{ldescription}
\item[\code{TExpH1}] numeric vector: the expectation of the test statistics under the alternative 
\item[\code{alpha}] a single numeric value: alpha-level, defaults to 0.05 
\item[\code{corrmatH1}] a numeric matrix, the correlation matrix of the teststatistics under the alternative, must have same dimensions as length of \code{TExpH1}  
\item[\code{alternative}] a single character string, specifying the alternative, one of "two.sided", "less", or "greater"  
\end{ldescription}
\end{Arguments}
\begin{Details}\relax
Any-pair-power is calcualted, the power of a Union-Intersection-Test.

The probability, that any teststatistic exceeds the critical value is calculated from a central multivariate standard normal distribution.
The appropriateness of the result strongly depends on the assumptions, that teststatistics are truely following a standard normal distributions,
which might not be the case for small sample sizes.
\end{Details}
\begin{Value}
A single numeric value, the approximative power.
\end{Value}
\begin{Author}\relax
Frank Schaarschmidt
\end{Author}
\begin{Examples}
\begin{ExampleCode}
\end{ExampleCode}
\end{Examples}

