\HeaderA{scalechoice}{Only for internal use.}{scalechoice}
\keyword{misc}{scalechoice}
\begin{Description}\relax
How to define scale parameters of the Weibull distribution in the poly-k-setting
\end{Description}
\begin{Usage}
\begin{verbatim}
scalechoice(shape = 3, pm = 0.2, t = 1)
\end{verbatim}
\end{Usage}
\begin{Arguments}
\begin{ldescription}
\item[\code{shape}] shape parameter (usually shape=3) 
\item[\code{pm}] a numeric value (0<p<1), probability that an individual dies before time t 
\item[\code{t}] a numeric value, time t 
\end{ldescription}
\end{Arguments}
\begin{Value}
a single numeric value, giving the scale parameter needed to fulfill pm(t) for the current definition of the Weibull disstribution.
\end{Value}
\begin{Note}\relax
Scale parameter is defined differently than that in Peddada (2005)
\end{Note}
\begin{Author}\relax
Frank Schaarschmidt
\end{Author}
\begin{References}\relax
\end{References}
\begin{SeeAlso}\relax
pweibull
\end{SeeAlso}
\begin{Examples}
\begin{ExampleCode}
\end{ExampleCode}
\end{Examples}

